\documentclass[a4paper, twoside,11pt]{report}
\usepackage[utf8]{inputenc}
\usepackage{amsmath}
\usepackage{amssymb}
\usepackage{arydshln}
\usepackage[a4paper,tmargin=2cm,bmargin=2cm,lmargin=2.5cm,rmargin=2.5cm]{geometry}
%\setlength{\topmargin}{0cm}
%\setlength{\textwidth}{6.75in}
%\setlength{\textheight}{9.25in}
%\setlength{\oddsidemargin}{-.25in}
%\setlength{\evensidemargin}{-.25in}
\linespread{1.5}
%\pagestyle{fancy}

\title{Complementos de Álgebra Linear}

\author{Cláudio Henriques
\\Fábio Henriques
\\Corretor: Micael Santos
\\
\\Universidade de Aveiro}

\date{2015/2016}

\begin{document}

\maketitle

\large\textbf{Proposta de resolução - Folha 1: Subespaços invariantes}
% -----------------------------------------------------------------
\vskip 0.25in
% -----------------------------------------------------------------
\makelettertitle

\begin{enumerate}

\item $\varphi \colon V \to V \quad \{ u\in V\colon\varphi(u)=\lambda u\} = U_\lambda$
\\Pretendemos provar que $u_\lambda$ é $\varphi$-invariante, isto é, provar:
\\(1) $U_\lambda$ é um subespaço; 
\\(2) $\varphi(U_\lambda)\subset U_\lambda$; 
\\ 
\\ (1): $U_\lambda$ é um subespaço vectorial, uma vez que é um subespaço associado ao valor próprio $ \lambda$. 
\\ (2): $\varphi(u) = \lambda u \in U_\lambda$ pois $U_\lambda$ é um subspaço vectorial e consequentemente é fechado para a multiplicação por um escalar.

% -----------------------------------------------------------------
\vskip 0.25in
% -----------------------------------------------------------------

\item $\varphi\colon V \to V \quad \forall x\in\setminus\{0\}$

\\- Hipótese: $x$ é um vector próprio. ($\varphi (x) = \lambda x$);
\\- Tese: $S=\langle x \rangle $ é $\varphi$-invariante.

\\ Por hipótese, $ \varphi (x)=\lambda x $, como $ S = \langle x \rangle,\quad \lambda x \in S \implies \varphi (x) \in S$. Logo, como S é um subespaço e $ \varphi(x) \in S$, S é $ \varphi$-invariante.

\\- Hipótese: $S=\langle x \rangle $ é $\varphi$-invariante, $ \varphi (S) \subset S$;
\\- Tese: $x$ é um vector próprio. ($\varphi (x) = \lambda x$).
\\Por hipótese, $ \varphi (x) \in S \implies \varphi (x) = \alpha x $, pois $x$ é um gerador de $S$. Nestas condições verificamos que $ \alpha \equiv \lambda$ e consequentemente $x$ é um vector próprio associado a $\lambda$. 

% -----------------------------------------------------------------
\vskip 0.25in
% -----------------------------------------------------------------

\item $\varphi \colon V \to V \quad \colon \quad x_1\dotso x_k$ são vetores próprios de $\varphi$
\\Pretende-se provar que o subespaço gerado pelos vectores próprios é $\varphi$-invariante.
\\\[
    \left.\begin{matrix}
    & \varphi(x_1) = \lambda x_1 \in V \\
    & \vdots \\
    & \varphi(x_k) = \lambda x_k \in V
    \end{matrix}\right\}
\textrm{pois V é fechado para o produto por um escalar.}}\]

\\$\therefore V$ satisfaz (1) e (2) (ver questão (1)), logo é $\varphi$-invariante.

% -----------------------------------------------------------------
\vskip 0.25in
% -----------------------------------------------------------------

\item $\varphi \colon V \to V\quad$ $u$, $w$ são $\varphi$-invariantes.
    \begin{enumerate}
        \item
        $\varphi(u\cap w) \subseteq \varphi(u) \subseteq u$
        \\$\varphi(u\cap w) \subseteq \varphi(w) \subseteq w$
        \\$\therefore \varphi(u\cap w) \subseteq u \cap w$
        \item
        \\$\varphi(u+w) = \varphi(u)+\varphi(w) \subseteq u+w$
    \end{enumerate}
\\Logo $u \cap w$ e $u+w$ são subespaços $\varphi$-invariantes de $V.$
     
% -----------------------------------------------------------------
\vskip 0.25in
% -----------------------------------------------------------------

\item

% -----------------------------------------------------------------
\vskip 0.25in
% -----------------------------------------------------------------

\item  $\varphi\colon  	\mathbb{R}^2 \to  \mathbb{R}^2$
\\    $M(\varphi, B)= \begin{bmatrix}
             2 & -5\\
             1 & -2\\
           \end{bmatrix}$
\\    $\begin{vmatrix}
        A - \lambda I\\
        \end{vmatrix} = \begin{vmatrix}
                        2-\lambda & -5\\
                        1 & -2-\lambda \\
                        \end{vmatrix} = (2-\lambda)(-2-\lambda)-(-5) =\lambda^2 +1
                        \\ \lambda^2 +1 =0 \Leftrightarrow \lambda = i \vee \lambda = -i$
                        \\$ \varphi$ não possui valores próprios.
\\ $\therefore$ Assim sendo, concluímos que não existem subespaços $\varphi$-invariantes não trivias.

% -----------------------------------------------------------------
\vskip 0.25in
% -----------------------------------------------------------------

\item  $\varphi\colon  	\mathbb{C}^2 \to  \mathbb{C}^2$
\\   $M(\varphi, B)= \begin{bmatrix}
             0 & 4\\
             -1 & 0\\
           \end{bmatrix}=A$
           \\
\\ $\cdot$ Calcular os valores próprios:
\\    $\begin{vmatrix}
        A - \lambda I\\
        \end{vmatrix} = \begin{vmatrix}
                        -\lambda & 4\\
                        -1 & -\lambda \\
                        \end{vmatrix} = \lambda^2 + 4$ 
                        \\$\lambda^2 = -4 \Leftrightarrow \lambda = -2i \vee \lambda = 2i$
\\
\\ Para $\lambda = -2i$
\\ $[A-\lambda I]\hat x=\hat 0 \Leftrightarrow \begin{bmatrix}
                                                 2i & 4\\
                                                -1 & 2i \\
                                                 \end{bmatrix}
\begin{bmatrix}
x\\
y\\
\end{bmatrix}
=\begin{bmatrix}
0\\
0\\
\end{bmatrix} \Leftrightarrow \begin{bmatrix}
                                    (2i)x + 4y\\
                                    -x + (2i)y\\
                                \end{bmatrix}
=\begin{bmatrix}
0\\
0\\
\end{bmatrix} 
\\\left\{\begin{matrix}
 2ix + 4y = 0 \\ 
 -x + 2iy = 0 
\end{matrix}\right.
\Leftrightarrow \left\{\begin{matrix}
 x = 2iy
\end{matrix}\right.$
\\
\\$U_-_2_i = \{(x, y) \in \mathbb{C}^2 \colon x = 2iy\} = \{(2iy, y),\;y \in R\} & \quad  U_-_2_i = \langle (2i, 1) \rangle $
\\
\\ Para $\lambda = 2i$
\\ $[A-\lambda I]\hat x=\hat 0 \Leftrightarrow \begin{bmatrix}
                                                 -2i & 4\\
                                                -1 & -2i \\
                                                 \end{bmatrix}
\begin{bmatrix}
x\\
y\\
\end{bmatrix}
=\begin{bmatrix}
0\\
0\\
\end{bmatrix} \Leftrightarrow \begin{bmatrix}
                                    (-2i)x + 4y\\
                                    -x + (-2i)y\\
                                \end{bmatrix}
=\begin{bmatrix}
0\\
0\\
\end{bmatrix} 
\\\left\{\begin{matrix}
 -2ix + 4y = 0 \\ 
 -x - 2iy = 0 
\end{matrix}\right.
\Leftrightarrow \left\{\begin{matrix}
 x = -2iy
\end{matrix}\right.$
\\
\\$U_2_i = \{(x, y) \in \mathbb{C}^2 \colon x = $ -$2iy\} = \{($-$2iy, y),\;y \in R\} & \quad  U_2_i = \langle ($-$2i, 1) \rangle $

% -----------------------------------------------------------------
\vskip 0.25in
% -----------------------------------------------------------------

\item
$B_V=(e_1, e_2)$, $\varphi \in End(V)$
\\$\varphi(e_1)=e_1 \quad e \quad \varphi(e_2)=e_1+e_2$
    \begin{enumerate}
        \item $M(\varphi, B)= \begin{bmatrix}
             1 & 1\\
             0 & 1\\
           \end{bmatrix} = A$
\\
\\ $\cdot$ Calcular os valores próprios:
            \\$\begin{vmatrix}
            A - \lambda I\\
            \end{vmatrix}=
           \begin{vmatrix}
           1-\lambda & 1\\
           0 & 1-\lambda\\
           \end{vmatrix}=(1-\lambda) (1-\lambda) = (1-\lambda)^2
           \\(1-\lambda)^2 = 0 \Leftrightarrow \lambda=1  $
           \\
           \\Para $\lambda = 1$
           \\$(A-1I_2)=\begin{bmatrix}
                        1 & 0\\
                        0 & 0\\
                        \end{bmatrix}
                        \begin{bmatrix}
                        x\\
                        y\\
                        \end{bmatrix} = \begin{bmatrix}
                        0\\
                        0\\
                        \end{bmatrix}\Leftrightarrow y=0$
                        \\$U_1 = \langle(1,\, 0)\rangle
                        \\ \therefore$ Logo $U_1=\langle(1,0)\rangle$ é um subespaço $\varphi$-invariante. Não existem mais subespaços $\varphi$-invariantes de $V$ não triviais.
        \item  Como $m_g (1)=1$ e $m_a (1) =2$, concluímos que não existem mais subespaços próprios e consequentemente fica assim impossível decompor $V$ numa soma direta de dois subespaços $\varphi$-invariantes não triviais (visto que nestas condições apenas existe um $\lambda$).
            \end{enumerate}
           
% -----------------------------------------------------------------
\vskip 0.25in
% -----------------------------------------------------------------

\item $\varphi \in End ($\mathbb{R}^2)
\\\varphi(x, y) = (x, -y)\quad\varphi(1,0) = (1,0)\quad\varphi(0,1) = (0,-1)
\\$M(\varphi, B_c)= \begin{bmatrix}
             1 & 0\\
             0 & -1\\
           \end{bmatrix}$
\\$B_c ((1,0),(0,-1))
\\
\\ \cdot$ Calcular os valores próprios:
\\    $\begin{vmatrix}
        A - \lambda I\\
        \end{vmatrix} = \begin{vmatrix}
                        1 -\lambda & 0\\
                        0 & -1-\lambda \\
                        \end{vmatrix} = \lambda^2 -1$ 
                        \\$\lambda^2 = 1 \Leftrightarrow \lambda = -1 \vee \lambda = 1$
\\                        
\\ Para $\lambda = -1$
\\ $[A-\lambda I]\hat x=\hat 0 \Leftrightarrow \begin{bmatrix}
                                                 2 & 0\\
                                                0 & 0 \\
                                                 \end{bmatrix}
\begin{bmatrix}
x\\
y\\
\end{bmatrix}
=\begin{bmatrix}
0\\
0\\
\end{bmatrix} \Leftrightarrow \begin{bmatrix}
                                    2x + 0\\
                                    0\\
                                \end{bmatrix}
=\begin{bmatrix}
0\\
0\\
\end{bmatrix} 
\\\left\{\begin{matrix}
 x = 0 \\ 
\end{matrix}\right.
\\U_{-1} = \{(x, y) \in \mathbb{R}^2 \colon x = 0\} = \{(0, y),\;y \in R\} & \quad  U_{-1} = \langle (0, 1) \rangle $
\\
\\ Para $\lambda = 1$
\\ $[A-\lambda I]\hat x=\hat 0 \Leftrightarrow \begin{bmatrix}
                                                 0 & 0\\
                                                0 & -2 \\
                                                 \end{bmatrix}
\begin{bmatrix}
x\\
y\\
\end{bmatrix}
=\begin{bmatrix}
0\\
0\\
\end{bmatrix} \Leftrightarrow \begin{bmatrix}
                                    0\\
                                    -2y + 0\\
                                \end{bmatrix}
=\begin{bmatrix}
0\\
0\\
\end{bmatrix} 
\\\left\{\begin{matrix}
 y = 0 \\ 
\end{matrix}\right.
\\U_{1} = \{(x, y) \in \mathbb{R}^2 \colon y = 0\} = \{(x, 0),\;x \in R\} & \quad  U_{1} = \langle (1, 0) \rangle $
\\Visto que o subespaço próprio de $\varphi$ associado ao valor próprio $\lambda$ é $\varphi$-invariante, então $U_{-1}=\langle(0,1)\rangle$ e $U_1=\large(1,0)\rangle$ são dois subespaços não triviais de $\varphi$.

% -----------------------------------------------------------------
\vskip 0.25in
% -----------------------------------------------------------------

\item 
    \begin{enumerate}
    \item $\varphi (1,2,1) = (2,1,-1) = 2 \times (1,2,1) - 3\times (0,1,1) \in S$
\\$\varphi(0,1,1) = (1,1,0) = 1 \times (1,2,1) - 1 \times (0,1,1) \in S$
\\Logo $S$ é $\varphi$-invariante. Vamos então criar uma base de $\mathbb{R}^3$ por completação da base de $S$.
\\Por exemplo $B=\langle(1,2,1),(0,1,1),(0,0,1)\rangle$
\\$\varphi(0,0,1) = (0,1,1) = 0 \times (1,2,1) + 1 \times (0,1,1) + 0 \times (0,0,1)$
\\Logo $M(\varphi, B)=\begin{bmatrix}
                        2 & 1 & 0\\
                        -3 & -1 & 1\\
                        0 & 0 & 0\\
                        \end{bmatrix}$
                        
    \item 
    $(x,y,z) = \alpha (1,0,0) + \beta(0,1,0) + \gamma (0,0,1) \Leftrightarrow \begin{cases}
    x = \alpha\\
    y = \beta\\
    z = \gamma
    \end{cases}$
    \\$\varphi (x,y,z) = x\varphi(1,0,0) + y\varphi(0,1,0) + z\varphi(0,0,1) \Leftrightarrow \varphi(x,y,z) = (x+y, x+2y + z, x+3y+2z)$
    \\$\varphi (1,-1,1) = (0,0,0) = 0 \times (1,-1,1) + 0\times (3,0,1) \in S$
\\$\varphi(3,0,1) = (3,4,5) \not \in S.
\\$\therefore$ Logo $S$ não é $\varphi$-invariante.
    \end{enumerate}

% -----------------------------------------------------------------
\vskip 0.25in
% -----------------------------------------------------------------

\item \\$A= \begin{bmatrix}
             1 & -1\\
             1 & 0\\
           \end{bmatrix}$
\\$\varphi(X) = AX$
    \begin{enumerate}
        \item $S=\langle I_2, A\rangle \Leftrightarrow
        S=\langle\begin{pmatrix}
        \begin{bmatrix}
             1 & -1\\
             1 & 0\\
           \end{bmatrix},
           \begin{bmatrix}
            1 & -1\\
            1 & 0\\
            \end{bmatrix}\end{pmatrix}\rangle
        \\\varphi(I_2) = A \in S
        \\\varphi(A) = AA = A^2 = \begin{bmatrix}
                            0 & -1\\
                            1 & -1\\
                            \end{bmatrix} = A - I_2 \in S$
        \\$\therefore$ Logo, $S$ é um subespaço $\varphi$-invariante.
        \item $\varphi(X) = \lambda X
        \\\varphi(I)=A\neq\lambda I_2$\textrm{, logo $S$ não é um subespaço próprio de $\varphi$ associado a um valor próprio.}
        \item $B=(I_2,A,E_{11},E_{12})$
\\Para $B$ ser uma base de $M_{2\times2}(\mathbb{R})$
\\$\alpha I_2 + \beta A + \gamma E_{11} + \delta E_{12} = 0
\\\begin{bmatrix}
    \alpha & 0\\
    0 & \alpha\\
    \end{bmatrix} + \begin{bmatrix}
    \beta & -\beta\\
    \beta & 0\\
    \end{bmatrix} + \begin{bmatrix}
    \gamma & 0\\
    0 & 0\\
    \end{bmatrix} + \begin{bmatrix}
    0 & \delta\\
    0 & 0\\
    \end{bmatrix} = \begin{bmatrix}
    0 & 0\\
    0 & 0\\
    \end{bmatrix} \Leftrightarrow \begin{bmatrix}
    \alpha + \beta + \gamma & -\beta + \delta\\
    \beta & \alpha\\
    \end{bmatrix}= \begin{bmatrix}
    0 & 0\\
    0 & 0\\
    \end{bmatrix} \Leftrightarrow \begin{cases}
    \alpha = 0\\
    \beta = 0\\
    \beta = \delta\\
    \alpha + \beta = \gamma
    \end{cases} \Leftrightarrow \begin{cases}
    \alpha = 0\\
    \beta = 0\\
    \gamma = 0\\
    \delta = 0\\
    \end{cases}$
    \textrm{O conjunto é linearmente independente com 4 elementos e dim $ M_{2\times2}(\mathbb{R})=4$, logo o conjunto B é uma base de $M_{2\times2}(\mathbb{R})}.$
    \item$\varphi(I_2)=(0,1,0,0)_B
    \\\varphi(A)=(-1,1,0,0)_B
    \\\varphi(E_{11})=\begin{bmatrix}
                        1 & 0\\
                        1 & 0\\
                        \end{bmatrix} = 0I_2 + 1A + 0E_{11} + 1E_{12} = (0,1,0,1)
    \\\varphi(E_{12})=\begin{bmatrix}
                        0 & 1\\
                        0 & 1\\
                        \end{bmatrix} = 1I_2 + 0A - 1E_{11} + 1E_{12} = (1,0,-1,1)
    \\M(\varphi,B)=\begin{bmatrix}
                \begin{tabular}{cc:cc}
                0 & -1 & 0 & 1\\
                1 & 1 & 1 & 0 \\ \hdashline
                0 & 0 & 0 & -1\\
                0 & 0 & 1 & 1\\
                \end{tabular}
                \end{bmatrix} \textrm{matriz triangular superior por blocos}$
    
    \item $\varphi(E_{11}) = \begin{bmatrix}
    1 & 0\\
    1 & 0\\
    \end{bmatrix} = E_{11} + E_{21} \in S
    \\\varphi(E_{21}) = \begin{bmatrix}
    1 & -1\\
    1 & 0\\
    \end{bmatrix} \begin{bmatrix}
    0 & 0\\
    1 & 0\\
    \end{bmatrix} = \begin{bmatrix}
    -1 & 0\\
    0 & 0\\
    \end{bmatrix} = -E_{11} \in S
    $\\\textrm{Falta mostrar que $T$ é complementar de $S$, isto é, que $S \oplus T = M_{2\times2}\mathbb{R}$. Para isso mostramos que}
    \\$B' = B_S \cup B_T = (I_2, A, E_{11}, E_{21})$
    \item $B' = (I_2, A, E_{11}, E_{21})$
    \\$M(\varphi;B')=\begin{bmatrix}
                \begin{tabular}{cc:cc}
                0 & -1 & 0 & 0\\
                1 & 1 & 0 & 0 \\ \hdashline
                0 & 0 & 1 & -1\\
                0 & 0 & 1 & 0\\
                \end{tabular}
                \end{bmatrix} $\textrm{matriz diagonal por blocos}$
    
    \end{enumerate}
    
% -----------------------------------------------------------------
\vskip 0.25in
% -----------------------------------------------------------------

\item $\begin{bmatrix}
        A & C\\
        0 & B\\
        \end{bmatrix}=\begin{bmatrix}
        I_m A + 0\times 0 & I_mC + 0 \times I_n\\
        0\times A + B\times 0 & 0 \times C + B\times I_n\\
        \end{bmatrix}=\begin{bmatrix}
        I_m & 0\\
        0 & B\\
        \end{bmatrix}
        \begin{bmatrix}
        A & C\\
        0 & I_n\\
        \end{bmatrix}
        \\
        \\\begin{vmatrix}
        A & C\\
        0 & B\\
        \end{vmatrix}=\begin{vmatrix}
        I_m & 0\\
        0 & B\\
        \end{vmatrix}\times \begin{vmatrix}
        A & C\\
        0 & I_n\\
        \end{vmatrix}=det\begin{pmatrix}
        B\\
        \end{pmatrix}\times det\begin{pmatrix}
        A\\
        \end{pmatrix}
        \\
        \\\begin{bmatrix}
        A & 0\\
        C & B\\
        \end{bmatrix}=\begin{bmatrix}
        A & 0\\
        0 & I_n\\
        \end{bmatrix}\times \begin{bmatrix}
        I_m & 0\\
        C & B\\
        \end{bmatrix}
        \\
        \\\begin{vmatrix}
        A & 0\\
        C & B\\
        \end{vmatrix}=\begin{vmatrix}
        A & 0\\
        0 & I_n\\
        \end{vmatrix}\times \begin{vmatrix}
        I_m & 0\\
        C & B\\
        \end{vmatrix}=det\begin{pmatrix}
        A\\
        \end{pmatrix}\times det\begin{pmatrix}
        B\\
        \end{pmatrix}$
% -----------------------------------------------------------------
\vskip 0.25in
% -----------------------------------------------------------------

\item 
    \begin{enumerate}
        \item
    $S  = \{(x,y,0)\colon x,y \in \mathbb{R}\} = \langle(1,0,0),(0,1,0)\rangle$
    \\$T = \{(0,0,z)\colon z \in \mathbb{R}\} = \langle(0,0,1)\rangle$
    \\$A = \begin{bmatrix}
        \cos \theta & - \sin \theta & 0\\
        \sin \theta & \cos \theta & 0\\
        0 & 0 & 1\\
        \end{bmatrix}$
    \\\left.\begin{matrix}
    $\varphi(1,0,0) = (\cos\theta,\sin\theta,0) = \cos \theta(1,0,0) + \sin\theta(0,1,0)
    \\\varphi(0,1,0) = (-\sin\theta,\cos\theta,0)=-\sin\theta(1,0,0)+\cos\theta(0,1,0)
    \end{matrix}\right\} $\textrm{ $S$ é $\varphi$-invariante}
    \\\varphi(0,0,1) = (0,0,1) \rightarrow \textrm{ $T$ é $\varphi$-invariante}$
        \item \varphi|_S \in End(S) \quad$e$\quad \varphi|_T \in End(T)
    \\P_\varphi(t) = det(A-tI_3) = \begin{vmatrix}
                                    \cos \theta - t & -\sin \theta & 0\\
                                    \sin \theta & \cos \theta - t & 0\\
                                    0 & 0 & 1-t\\
                                    \end{vmatrix} =
    \\= [(\cos\theta - t) \times (\cos \theta - t) \times (1-t)]-[(-\sin \theta) \times (\sin \theta) \times (1-t)] = (1-t)\times[1+t^2-(2t\cos\theta)]=det(P_\varphi_{|S})\times det(P_\varphi_{|T})
    \\A=\begin{bmatrix}
        \cos\theta & -\sin \theta & 0\\
        \sin \theta & \cos \theta & 0\\ 
        0 & 0 & 1\\
        \end{bmatrix}$
    
\end{enumerate}

% -----------------------------------------------------------------
\vskip 0.25in
% -----------------------------------------------------------------

\item 
 \begin{enumerate}
     \item Para que $\langle(1,0,0,1),(0,1,1,-1)\rangle$ seja um subespaço $\varphi$-invariante de $\mathbb{R}^4$, então $\varphi(1,0,0,1) = \alpha(1,0,0,1) +\beta (0,1,1,-1) $ e $\varphi(0,1,1,-1) = \alpha(1,0,0,1) + \beta(0,1,1,-1) $.
        \\$\varphi(1,0,0,1) = A\times(1,0,0,1) =\begin{bmatrix}
                                            3&-1&-1&-2\\
                                            1&1&-1&-1\\
                                            1&0&0&-1\\
                                            0&-1&1&1\\
                                            \end{bmatrix}\times \begin{bmatrix}
                                            1\\
                                            0\\
                                            0\\
                                            1\\
                                            \end{bmatrix}=\begin{bmatrix}
                                            1\\
                                            0\\
                                            0\\
                                            1\\
                                            \end{bmatrix}
    \\\varphi(0,1,1,-1) = A\times(0,1,1,-1)=\begin{bmatrix}
                                            3 & -1 & -1 & -2\\
                                            1 & 1 & -1 & -1\\
                                            1 & 0 & 0 & -1\\
                                            0 & -1 & 1 & 1\\
                                            \end{bmatrix}\times \begin{bmatrix}
                                            0\\
                                            1\\
                                            1\\
                                            -1\\
                                            \end{bmatrix}=\begin{bmatrix}
                                            0\\
                                            1\\
                                            1\\
                                            -1\\
                                            \end{bmatrix}
\\ \varphi(1,0,0,1) = 1\times (1,0,0,1)
    \\ \varphi(0,1,1,-1) = 1\times (0,1,1,-1)
    \\ \therefore$ Logo, $\langle(1,0,0,1),(0,1,1,-1)\rangle$ é um subespaço $\varphi$-invariante de $\mathbb{R}^4.$
    \item $\varphi(1,0,0,1) = (1,0,0,1)
    \\ \varphi(0,1,1,-1) = (0,1,1,-1)$
    \\Usemos os seguintes vectores para completar a base $B'$: $(0,0,1,0)$ e $(0,0,0,1)$
    \\$\varphi(0,0,1,0) = A\times(0,0,1,0) = \begin{bmatrix}
                                            3 & -1 & -1 & -2\\
                                            1 & 1 & -1 & -1\\
                                            1 & 0 & 0 & -1\\
                                            0 & -1 & 1 & 1\\
                                            \end{bmatrix} \times \begin{bmatrix}
                                            0\\
                                            0\\
                                            1\\
                                            0\\
                                            \end{bmatrix} = \begin{bmatrix}
                                            -1\\
                                            -1\\
                                            0\\
                                            1\\
                                            \end{bmatrix}
     \\(-1,-1,0,1)=\alpha(1,0,0,1)+\beta(0,1,1,-1)+\gamma(0,0,1,0) +\delta(0,0,0,1)
    \\\begin{cases}
       -1 = \alpha\\
        -1 = \beta\\
        0 = \beta + \gamma\\
        1=\alpha-\beta+\delta\\
        \end{cases} \Leftrightarrow \begin{cases}
        \alpha = -1\\
        \beta = -1\\
        \gamma = 1 \\
        \delta = 1\\
        \end{cases}$
    \\$\varphi(0,0,0,1) = A \times(0,0,0,1) =  \begin{bmatrix}
                                            3 & -1 & -1 & -2\\
                                            1 & 1 & -1 & -1\\
                                            1 & 0 & 0 & -1\\
                                            0 & -1 & 1 & 1\\
                                            \end{bmatrix} \times \begin{bmatrix}
                                            0\\
                                            0\\
                                            0\\
                                            1\\
                                            \end{bmatrix} = \begin{bmatrix}
                                            -2\\
                                            -1\\
                                            -1\\
                                            1\\
                                            \end{bmatrix}
        \\(-2,-1,-1,1)=\alpha(1,0,0,1)+\beta(0,1,1,-1)+\gamma(0,0,1,0) +\delta(0,0,0,1)
        \\\begin{cases}
       -2 = \alpha\\
        -1 = \beta\\
        -1 = \beta + \gamma\\
        1=\alpha-\beta+\delta\\
        \end{cases} \Leftrightarrow \begin{cases}
        \alpha = -2\\
        \beta = -1\\
        \gamma = 0 \\
        \delta = 2\\
        \end{cases}$
        \\$M(\varphi, B')=\begin{bmatrix} \begin{tabular}{cc:cc}
         1 & 0 & -1 & -2\\
         0 & 1 & -1 & -1\\ \hdashline
         0 & 0 & 1 & 0\\
         0 & 0 & 1 & 2\\
        \end{tabular}\end{bmatrix}$
    \end{enumerate}


% -----------------------------------------------------------------
\vskip 0.25in
% -----------------------------------------------------------------


\item
    \begin{enumerate}
        \item $f(\begin{bmatrix}
                    1 & 0\\
                    1 & 0\\
                    \end{bmatrix})=\begin{bmatrix}
                    -1 & -2\\
                    1 & -2\\
                    \end{bmatrix}=1\times \begin{bmatrix}
                    1 & 0\\
                    1 & 0\\
                    \end{bmatrix}-2\times \begin{bmatrix}
                    1 & 1\\
                    0 & 1\\
                    \end{bmatrix}
                \\ f(\begin{bmatrix}
                    1 & 1\\
                    0 & 1\\
                    \end{bmatrix})=\begin{bmatrix}
                    1 & -1\\
                    2 & -1\\
                    \end{bmatrix}=2\times \begin{bmatrix}
                    1 & 0\\
                    1 & 0\\
                    \end{bmatrix}-1\times \begin{bmatrix}
                    1 & 1\\
                    0 & 1\\
                    \end{bmatrix}$
                \\ $\therefore$ Logo o subespaço $\langle\begin{bmatrix}
                    1 & 0\\
                    1 & 0\\
                    \end{bmatrix}, \begin{bmatrix}
                    1 & 1\\
                    0 & 1\\
                    \end{bmatrix}\rangle$ é $f$-invariante.
        \item Para determinar a matriz $M(\varphi, B')$, iremos usar as seguintes matrizes:
        $\begin{bmatrix}
                0 & 0\\
                0 & 1\\
                \end{bmatrix}$, $\begin{bmatrix}
                1 & 1\\
                1 & 1\\
                \end{bmatrix}$, $\begin{bmatrix}
                1 & 0\\
                1 & 0\\
                \end{bmatrix}$ e $\begin{bmatrix}
                1 & 1\\
                0 & 1\\
                \end{bmatrix}$.
        \\$\varphi(E_4_4)=\begin{bmatrix}
                            1 & 0\\
                            0 & 0\\
                            \end{bmatrix}$
        \\$\begin{bmatrix}
                    1 & 0\\
                    0 & 0\\
                    \end{bmatrix} = \alpha \begin{bmatrix}
                            0 & 0\\
                            0 & 1\\
                            \end{bmatrix} +\beta \begin{bmatrix}
                            1 & 1\\
                            1 & 1\\
                            \end{bmatrix} + \gamma \begin{bmatrix}
                            1 & 1\\
                            0 & 1\\
                            \end{bmatrix} + \lambda \begin{bmatrix}
                            1 & 0\\
                            1 & 0\\
                            \end{bmatrix}$
        \\$\begin{cases}
                \alpha = 0\\
                \beta = -1\\
                \gamma = 1\\
                \lambda = 1\\
                \end{cases} $
        \\$\varphi(\begin{bmatrix}
                            1 & 1\\
                            1 & 1\\
                            \end{bmatrix})=\begin{bmatrix}
                            1 & -4\\
                            2 & -3\\
                            \end{bmatrix}$
        \\$\begin{bmatrix}
                    1 & 0\\
                    0 & 0\\
                    \end{bmatrix} = \alpha \begin{bmatrix}
                            0 & 0\\
                            0 & 1\\
                            \end{bmatrix} +\beta \begin{bmatrix}
                            1 & 1\\
                            1 & 1\\
                            \end{bmatrix} + \gamma \begin{bmatrix}
                            1 & 1\\
                            0 & 1\\
                            \end{bmatrix} + \lambda \begin{bmatrix}
                            1 & 0\\
                            1 & 0\\
                            \end{bmatrix}$
        \\$\begin{cases}
                \alpha = 1\\
                \beta = -2\\
                \gamma = -2\\
                \lambda = 4\\
                \end{cases} $
        \\$M(\varphi, B') = \begin{bmatrix} \begin{tabular}{cc:cc}
                            0 & 1 & 0 & 0\\
                            -1 & -2 & 0 & 0\\ \hdashline
                            1 & -2 & -1 & -2\\
                            1 & 4 & 2 & 1\\
                            \end{tabular}\end{bmatrix}$ 
        
                                    
    \end{enumerate}
    
% -----------------------------------------------------------------
\vskip 0.25in
% -----------------------------------------------------------------

\item 
 \begin{enumerate}
        \item \begin{enumerate}
            \item$\varphi\colon\mathbb{R}^3\to\mathbb{R}^3\quad\varphi(x,y,z) = (x+3y,2x-y+z,x-z)$, $v=(1,1,1)$, $Z(\varphi, v)=\langle v, \varphi(v), ...\rangle$
            \\$\varphi(v)=\varphi(1,1,1)=(4,2,0) \neq \alpha(1,1,1),\quad \forall_\alpha\in\mathbb{R}
            \\\varphi^2(v)=\varphi(4,2,0)=(10,6,4) \neq \alpha(1,1,1)+\beta(4,2,0), ,\quad \forall_{\alpha,\beta}\in\mathbb{R}
            \\\varphi^3(v)=\varphi(10,6,4)=(28,18,6) = \alpha(1,1,1)+\beta(4,2,0)+\gamma(10,6,4)$
            $\\(28,18,6)=\alpha\times (1,1,1) + \beta \times (4,2,0)+\gamma\times (10,6,4)
            \Leftrightarrow(28,18,6)=(\alpha,\alpha,\alpha) + (4\beta,2\beta,0)+(10\gamma,6\gamma,4\gamma)$
            \\\begin{cases}
                $28 = \alpha+4\beta+10\gamma\\
                18 = \alpha+2\beta+6\gamma\\
                6 = \alpha+0\beta+4\gamma\\
            \end{cases}
            =\begin{cases}
                \alpha = 10\\
                \beta = 7\\
                \gamma = -1\\
            \end{cases}$
            \\$Z(\varphi,v)=\langle(1,1,1),(4,2,0),(10,6,4)\rangle$
            \item$\varphi\colon\mathbb{R}^2\to\mathbb{R}^2\quad\varphi(x,y)=(-y,x)$, $v=(a,b)$, $Z(\varphi,v)=\langle v,\varphi(v),...\rangle$
            \\$\varphi(v)=\varphi(a,b)=(-b,a)\neq\alpha(a,b)
            \\\varphi^2(v)=\varphi(-b,a)=(-a,-b)=-v$
            \\$Z(\varphi,v)=\langle(a,b),(-b,a)\rangle$
            \item$\varphi\colon\mathbb{R}^3\to\mathbb{R}^3\quad\varphi(x,y,z)=(x+z,x+y,z)$, $v=(1,0,0)$, $Z(\varphi,v)=\langle v,\varphi(v),...\rangle
            \\\varphi(v)=\varphi(1,0,0)=(1,1,0)\neq\alpha(1,0,0),\quad \forall_\alpha\in\mathbb{R}
            \\\varphi^2(v)=\varphi(1,1,0)=(1,2,0)=-1\times(1,0,0)+2\times(1,1,0)
            \\Z(\varphi,v)=\langle(1,0,0),(1,1,0)\rangle$
        \end{enumerate}
        \item\begin{enumerate}
            \item$B_*=((1,1,1),(4,2,0),(10,6,4))$
            \item$B_*=((a,b),(-b,a))$
            \item$B_*=((1,0,0),(1,1,0))$
        \end{enumerate}
        \item
        \begin{enumerate}
            \item
            $\\M(\varphi_{|v},B_*)=\begin{bmatrix}
            0 & 0 & 10\\
            1 & 0 & 7\\
            0 & 1 & -1\\
            \end{bmatrix}$
            \item$\varphi^2(v)=C_0\times v + C_1 \times \varphi(v)
            \\(-a,-b)=(a \times C_0, b \times C_0)+(-b \times C_1, a \times C_1)
            \\\begin{cases}
                -a = a\times C_0+(-b)\times C_1\\
                -b = b\times C_0+a\times C_1\\
            \end{cases} = \begin{cases}
                C_0= -1\\
                C_1 = 0\\
            \end{cases}
            \\M(\varphi_{|v},B_*)=\begin{bmatrix}
            0 & -1\\
            1 & 0\\
            \end{bmatrix}$
            \item Para completar a base $B_*$ iremos usar o vetor $(0,0,1)$
            \\$(0,0,0)=\alpha(0,0,1)+\beta(1,0,0)+\gamma(1,1,0)$
            \\$\begin{cases}
                \alpha=0\\
                \beta=0\\
                \gamma=0\\
            \end{cases}$
            \\$\varphi(0,0,1)=(1,0,1)
            \\(1,0,1)=\alpha(1,0,0)+\beta(1,1,0)+\gamma(0,0,1)
            \\\begin{cases}
                1=\alpha+\beta\\
                0=\beta\\
                1=\gamma\\
                \end{cases} = \begin{cases}
                                \alpha=1\\
                                \beta=0\\
                                \gamma=1\\
                                \end{cases}
            \\M(\varphi_{|v},B_*)=\begin{bmatrix}
            0 & -1 & 1\\
            2 & 1 & 0\\
            0 & 0 & 1\\
            \end{bmatrix}
            $
        \end{enumerate}
    \end{enumerate}

% -----------------------------------------------------------------
\end{enumerate}



\end{document}